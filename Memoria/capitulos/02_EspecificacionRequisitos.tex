\begin{titlepage}
	
\chapter{Especificación de requisitos}

\section{Objetivos}

A modo de resumen, los principales objetivos que se pretenden alcanzar con el producto software
son:

\begin{itemize}
	
	\item \textbf{OBJ 1}: El sistema conectará con la cuenta del usuario de la plataforma de trading en cuestión, para poder hacer operaciones y visualizar datos en tiempo real.
	\item \textbf{OBJ 2}: El sistema permitirá a los usuarios ver esquemas en tiempo real del mercado que se quiera visualizar.
	\item \textbf{OBJ 3}: El sistema permitirá a los usuarios hacer operaciones de compra y venta de manera manual en un mercado concreto, a través de la plataforma.
	\item \textbf{OBJ 4}: El sistema desarrollará varios modelos para predecir el comportamiento de los mercados.
	\item \textbf{OBJ 5}: El sistema permitirá a los usuarios hacer operaciones de compra y venta de manera automatizada en un periodo de tiempo y mercado concretos, eligiendo los modelos de predicción mencionados.
	\item \textbf{OBJ 6}: El sistema permitirá a los usuarios probar cada uno de los modelos en un periodo de tiempo fijo, a modo de backtesting.
	
\end{itemize}

\section{Descripción de los implicados y usuario final}

En esta aplicación, destacamos dos principales implicados: el administrador de la aplicación y el usuario final de la aplicación.

\begin{itemize}
	\item \textbf{Desarrollador del sistema}: La responsabilidad del implicado será la de realizar las distintas actividades de desarrollo de la aplicación: corregir errores o añadir nuevas features, entre otras actualizaciones, que garantizan el correcto funcionamiento del sistema y su mantenimiento.
	
	\item \textbf{Usuario de la aplicación}: Este implicado representa al cliente que usa la aplicación. Este implicado hace uso de la aplicación como usuario final.
\end{itemize}

\section{Requisitos Funcionales}

Descripción de los requisitos más importantes a nivel de funciones que debe incluir el sistema, realizando una clasificación en categorías, a cada uno de los requisitos se le ha asignado un código y un nombre, con el fin de identificarlos fácilmente a lo largo de todo el proyecto. 

\begin{itemize}
	
	\item \textbf{RF-1. Gestiones de la plataforma de trading.} 
	\begin{itemize}
		\item \textbf{RF-1.1. Login en la plataforma de trading.} El usuario de la aplicación podrá conectarse con su cuenta de trading, comercial o demo, para poder hacer el uso completo de la APP.
		\item \textbf{RF-1.2. Ver capital disponible.} El usuario podrá ver el capital disponible en su cuenta de trading.
		\item \textbf{RF-1.3. El usuario podrá ver información de las operaciones} realizadas y de operaciones que en ese momento aún no se han cerrado.
	\end{itemize}

	\item \textbf{RF-2. Procesado de datos.} El usuario de la aplicación podrá proporcionar datos de mercados que procesará el sistema y que servirán de \textit{input} para cada uno de los algoritmos de predicción.

	\item \textbf{RF-3. Visualización de datos.}
	\begin{itemize}
		\item \textbf{RF-3.1. Ver datos de mercado específico.} El usuario de la aplicación podrá ver información de precios de un mercado específico.
		\item \textbf{RF-3.2. Ver datos de mercado en rango de tiempo específico.} El usuario de la aplicación podrá ver información de precios entre dos fechas específicas.
		\item \textbf{RF-3.3. Ver datos de mercado con un marco de tiempo específico.} El usuario de la aplicación podrá ver información de precios en tiempo real o entre fechas con un marco de tiempo específico.
	\end{itemize}

	\item \textbf{RF-4. El usuario podrá realizar operaciones de compra y venta de forma manual}, en un mercado, con sus respectivos parámetros.

	\item \textbf{RF-5. El usuario podrá elegir un modelo y realizar trading algorítmico.} También podrá parametrizarlo según modelo y elegir tiempo en el que quiere dejar haciendo las operaciones automáticas. 

	\item \textbf{RF-6. El usuario podrá elegir un modelo y realizar trading algorítmico a modo de backtesting.} De esta forma podrá probar cada uno de los modelos en un mercado y periodo de tiempo prefijados.
	
\end{itemize}

\section{Requisitos No Funcionales}

\begin{itemize}
	\item \textbf{RNF-1}. La plataforma de trading que usará la aplicación será MetaTrader5.
	\item \textbf{RNF-2}. La aplicación permitirá el uso de cualquier bróker aceptado por MetaTrader5, teniendo el usuario que configurar los parámetros de comisión correspondientes.
	\item \textbf{RNF-3}. Para la visualización de datos, el usuario podrá elegir un marco de tiempo de entre entre  m1, m5, m15, m30, h1, h4, d1, w1 ó mn (minutos, horas, días, semanas o meses, respectivamente).
	\item \textbf{RNF-4}. La aplicación responderá a las peticiones de los usuarios en un tiempo determinado,
	mostrando un aviso de error si el tiempo de respuesta es superior al establecido.
	\item \textbf{RNF-5}. La aplicación deberá funcionar computadoras mediante navegador web.
	\item \textbf{RNF-6}. Para la implementación de la aplicación, se utilizará Python y su framework Django.
	\item \textbf{RNF-7}. La interfaz debe ser sencilla e intuitiva.
\end{itemize}

\section{Requisitos de Información}

\begin{itemize}
	\item \textbf{RI-1}. Los datos introducidos por el usuario tendrán formato \textit{csv}. Requisitos asociados: RF-2. 
\end{itemize}

\end{titlepage}
