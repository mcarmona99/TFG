\begin{titlepage}

\chapter{Introducción}

\section{Trading y otros conceptos}

El trading es una práctica aplicada a los mercados financieros. Realizar trading implica comprar o vender dentro de los distintos mercados financieros para vender o comprar de nuevo al cabo de un tiempo y obtener beneficio de la acción realizada. A grandes rasgos y con un ejemplo, realizar trading beneficiosamente podría ser comprar una acción de Apple por 100 dólares y venderla en cierto tiempo por 150 dólares, obteniendo por tanto un beneficio de 50 dólares. \newline

¿Cuándo se debe comprar o vender un activo? Los traders u operadores trabajan en torno a dos principales tipos de análisis:

\begin{itemize}

	\item \textbf{Análisis fundamental}: El análisis fundamental consiste en operar en base a las distintas noticias que ocurren en el mundo diariamente. Estas noticias sirve de fundamento para actuar de una forma u otra en el mercado.
	\item \textbf{Análisis técnico}: En este tipo de análisis, el operador o trader opera puramente en base al precio: su acción y movimientos. En este caso, no se tienen en cuenta las noticias. Es este el tipo de análisis interesante a la hora de automatizar. Esto ocurre porque en este método entran en juego cálculos matemáticos como medias, medias móviles, zonas de probabilidad estadística, etc. A partir de este análisis surge el trading algorítmico. \\
	Este análisis, aunque se pueda automatizar, no te asegura que el precio vaya en la dirección resultado del análisis. A pesar de esto, si trabajamos con probabilidades y obtenemos probabilidades de ganar superiores al 50%, obtendremos un sistema que a largo plazo conseguiría grandes beneficios.
	
\end{itemize}

\section{Trading algorítmico}

En pocas palabras, el Trading algorítmico es implementar un sistema de Trading que opere de forma automática.
	

\section{¿Qué es TradingAPP?}

TradingAPP es una aplicación web para realizar trading algorítmico basándose en técnicas de análisis de mercados financieros clásicas. La aplicación se desarrolla en el lenguaje de programación Python y usa el framework Django. \newline

El desarrollo principal de la aplicación se encuentra en la capacidad para comprar y vender de forma automática usando una cuenta comercial real tal y como lo haría un usuario humano. Estas operaciones se realizan según lo indique el algoritmo que elijamos, dentro de una lista de algoritmos que encontramos en la propia aplicación. \newline

Para hacer uso de la aplicación, primero deberemos iniciar sesión en la misma, con un usuario y contraseñas que conocemos de antemano y que nos proporciona el administrador. TradingAPP permite al usuario identificarse con su cuenta comercial o demo de MetaTrader5, que será la aplicación externa que realizará las compras y ventas en el mercado financiero seleccionado. \newline

Una vez un usuario está identificado y ha iniciado sesión en MT5, podrá escoger un algoritmo para realizar Trading automático en tiempo real o probar las técnicas a modo de Backtesting. La aplicación también proporciona la posibilidad de ver el histórico de operaciones realizado y el balance actual de la cuenta de MT5 en la que se ha identificado el usuario. \newline

Además de poder realizar operaciones de manera automática, la aplicación dispone de una interfaz propia para ver los datos de mercado en tiempo real o antiguos, utilizando gráficas interactivas.

\end{titlepage}
